%%%%%%%%%%%%%%%%%%%%%%%%%%%%%%%%%%%%%%%%%%%%%%%%%%%%%%%%%%%%%%%%%%%%%%%%%%%%%%%%
% Medium Length Graduate Curriculum Vitae
% LaTeX Template
% Version 1.2 (3/28/15)
%
% This template has been downloaded from:
% http://www.LaTeXTemplates.com
%
% Original author:
% Rensselaer Polytechnic Institute 
% (http://www.rpi.edu/dept/arc/training/latex/resumes/)
%
% Modified by:
% Daniel L Marks <xleafr@gmail.com> 3/28/2015
%
% Important note:
% This template requires the res.cls file to be in the same directory as the
% .tex file. The res.cls file provides the resume style used for structuring the
% document.
%
%%%%%%%%%%%%%%%%%%%%%%%%%%%%%%%%%%%%%%%%%%%%%%%%%%%%%%%%%%%%%%%%%%%%%%%%%%%%%%%%

%-------------------------------------------------------------------------------
%	PACKAGES AND OTHER DOCUMENT CONFIGURATIONS
%-------------------------------------------------------------------------------

%%%%%%%%%%%%%%%%%%%%%%%%%%%%%%%%%%%%%%%%%%%%%%%%%%%%%%%%%%%%%%%%%%%%%%%%%%%%%%%%
% You can have multiple style options the legal options ones are:
%
%   centered:	the name and address are centered at the top of the page 
%				(default)
%
%   line:		the name is the left with a horizontal line then the address to
%				the right
%
%   overlapped:	the section titles overlap the body text (default)
%
%   margin:		the section titles are to the left of the body text
%		
%   11pt:		use 11 point fonts instead of 10 point fonts
%
%   12pt:		use 12 point fonts instead of 10 point fonts
%
%%%%%%%%%%%%%%%%%%%%%%%%%%%%%%%%%%%%%%%%%%%%%%%%%%%%%%%%%%%%%%%%%%%%%%%%%%%%%%%%

\documentclass[margin]{res}  

% Default font is the helvetica postscript font
\usepackage{helvet}

% Increase text height
\textheight=700pt

\begin{document}

%-------------------------------------------------------------------------------
%	NAME AND ADDRESS SECTION
%-------------------------------------------------------------------------------
\name{Charles Dickens}

% Note that addresses can be used for other contact information:
% -phone numbers
% -email addresses
% -linked-in profile

\address{Cell: (808)286-0540}
\address{Email: charlieadickens@gmail.com}

% Uncomment to add a third address
%\address{Address 3 line 1\\Address 3 line 2\\Address 3 line 3}
%-------------------------------------------------------------------------------

\begin{resume}

%-------------------------------------------------------------------------------
%	EDUCATION SECTION
%-------------------------------------------------------------------------------
\section{EDUCATION}
\textbf{University of California Santa Cruz}, Santa Cruz, Ca.\\
{\sl PhD.}, Computer Science
\\
\textbf{University of Hawaii}, Manoa, Hi\\
{\sl B.S. with Summa Cum Laude}, \\ 
Computer Engineering and Mathematics Minor, May 2019
\hfill GPA: 
3.93

%-------------------------------------------------------------------------------

%-------------------------------------------------------------------------------
%	EXPERIENCE SECTION
%-------------------------------------------------------------------------------
% Modify the format of each position
\begin{format}
\title{l}\employer{r}\\
\dates{l}\location{r}\\
\body\\
\end{format}
%-------------------------------------------------------------------------------

\section{EXPERIENCE}
\employer{UCSC}
\location{Santa Cruz Ca.}
\dates{September 2019 - Present}
\title{\textbf{Teaching Assistant}}
\begin{position}
TA for UCSC CSE Introduction to Algorithms and Analysis
\begin{itemize}
\item Attended and presented lectures
\item Prepared sample solutions to homework problems for graders
\end{itemize}
\end{position}

\employer{Clari}
\location{Sunnyvale Ca.}
\dates{June 2019 - September 2019}
\title{\textbf{Data Science Intern}}
\begin{position}
Researched methods and applications for a new sales opportunity similarity metric.
\begin{itemize}
\item Developed and verified methods for measuring the similarity between deals.
\item Demonstrated applications of the metric to department executives.
\item Created a roadmap for scalable implementation of the feature.
\item Created a UI framework for future demos with REACT, SemanticUI, and D3.js.
\end{itemize}
\end{position}

\employer{University Health Partners of Hawaii}
\location{Honolulu Hi.}
\dates{April 2019 - June 2019}
\title{\textbf{Computer Engineer Intern}}
\begin{position}
Utilized H2O, MapR, and SQL to develop a pipeline for triaging documents for review. 
\begin{itemize}
\item Improved model performance by implementing procedures for hyperparameter optimization, model selection, and feature space dimension reduction.
\item Developed scalable pipelines for prediction and training.
\item Added functionality to extract human interpretable information as to why documents where marked for review . 
\end{itemize}
\end{position}

\employer{Hawaii Data Science Institute}
\location{Honolulu Hi.}
\dates{June 2018 - June 2019}
\title{\textbf{Data Science Fellow}}
\begin{position}
Contributed to data science products and applications across a range of disciplines.
\begin{itemize}
\item Wrote a course on data wrangling with Python and pandas using Jupyter. 
\item Planned and led workshops on Data Science topics.
\end{itemize}
\end{position}

\employer{Online Learning Academy}
\location{Honolulu Hi.}
\dates{December 2016 - August 2018}
\title{\textbf{Math and Science Tutor}}
\begin{position}
Tutored STEM related courses from grade school up to the sophomore college level.
\end{position}

%-------------------------------------------------------------------------------

%-------------------------------------------------------------------------------
%	PROJECTS SECTION
%-------------------------------------------------------------------------------
\section{PROJECTS}
\par
\textbf{Developer:} CSDL: Open Power Quality: \\
OPQ is an open source solution for distributed power quality data collection, analysis and visualization.
\begin{itemize}
\item First author of an accepted paper presented at The Ninth International Conference on Smart Grids, Green Communications and IT Energy-aware Technologies in Athens, Greece: IARIA ENERGY 2019.
\item Developed plugins to classify frequency disturbances and transients using state of the art digital signal processing and machine learning techniques.
\end{itemize}

\par
\textbf{Co-PI:} UH Big Data Lab and CSOI: The Polarization of Information:\\
How can we develop a graphical representation of related text based information found on the web? With this model, how can we measure the polarity between different schools of thought?
\begin{itemize}
\item Built scrapers to find relevant sources and utilized Social Network APIs to collect and store data in a MySQL database.
\item Implemented novel machine algorithms to model the data and identify clusters.
\item Project funded by the NSF Center for Science of Information.
\item On-boarded a team of 4 researchers from various Universities.
\end{itemize}

%-------------------------------------------------------------------------------

%-------------------------------------------------------------------------------
%	COMPUTER SKILLS SECTION
%-------------------------------------------------------------------------------
\section{COMPUTER\\SKILLS}

\textbf{Languages}: Python, R, C, C++, Java, Matlab, SQL, \LaTeX.
\\
\textbf{Web}: HTML, JavaScript, MongoDB, Meteor, React, Semantic UI
\\
\textbf{Applications}: Vi/Vim, JetBrains IDEs, Git/GitHub, VirtualBox

%-------------------------------------------------------------------------------

%-------------------------------------------------------------------------------
% Volunteer and non-Profit Work
%-------------------------------------------------------------------------------
% Modify the format of each position
\begin{format}
\title{l}\employer{r}\\
\dates{l}\location{r}\\
\body\\
\end{format}
%-------------------------------------------------------------------------------

\section{VOLUNTEER and NON-PROFIT WORK}
\employer{IEEE Infocom}
\location{Honolulu Hi.}
\dates{2018}
\title{\textbf{Student Volunteer}}
\begin{position}
Aided presenters and event organizers at the IEEE Infocom conference.
\end{position}

\employer{UHM COE}
\location{Honolulu Hi.}
\dates{Jun 2017 - August 2017}
\title{\textbf{Mentor}}
\begin{position}
Taught students fundamentals data science skills and guided them through a project over the 6-week course.
\end{position}

%-------------------------------------------------------------------------------

%-------------------------------------------------------------------------------
% Trainings and Workshops
%-------------------------------------------------------------------------------
% Modify the format of each position
\begin{format}
\title{l}\employer{r}\\
\dates{l}\location{r}\\
\body\\
\end{format}
%-------------------------------------------------------------------------------

\section{TRAININGS and WORKSHOPS}
\employer{Intelligent Systems for Geosciences}
\location{UH Hilo}
\dates{August 2018}
\title{\textbf{Researcher}}
\begin{position}
Worked with domain scientists to develop sensors and propose new directions for research and collaboration.
\begin{itemize}
\item Built a dual spectrum imaging sensor using a Raspberry Pi to periodically capture photos of the active fissure. Stored the data on AWS cloud storage.
\end{itemize}
\end{position}

\employer{Data Science and Interdisciplinary Teams}
\location{Purdue University}
\dates{June 2018}
\title{\textbf{Team Leader}}
\begin{position}
Invited as team leader for a group of 5 students from Universities across the country and at levels ranging from Undergraduate to PhD candidates.
\end{position}

%-------------------------------------------------------------------------------
%Awards
%-------------------------------------------------------------------------------
\section{AWARDS}

\textbf{UCSC Regents Fellowship}
\\
\textbf{NSF CSoI Channels Scholar REU Award}
\\
\textbf{Dean’s List 10 Semesters}


\end{resume}
\end{document}